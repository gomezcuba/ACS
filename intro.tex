\documentclass[xcolor=dvipsnames,aspectratio=169]{beamer}

% INCLUSIÓN DOS PAQUETES IMPRESCINDIBLES DE IDIOMA E CODIFICACIÓN DE CARACTERE.
\usepackage[T1]{fontenc}
\usepackage[english]{babel}
\usepackage[utf8]{inputenc}
\usepackage{csquotes}

%ACRONYMS para engadir un glosario de acronimos automatizado
% \usepackage[acronyms,nonumberlist,nopostdot,nomain,nogroupskip]{glossaries}
% \input{./acronyms.tex}

% PAQUETES PARA FIGURAS E GRAFICOS
\usepackage{graphicx}
%   \usepackage[pdftex]{graphicx}
  \usepackage{epstopdf}
   \graphicspath{{./img/}}
  % and their extensions so you won't have to specify these with
  % every instance of \includegraphics
   \DeclareGraphicsExtensions{.eps,.pdf,.png,.jpg}   
\usepackage{subfigure}
\usepackage{caption}

%Tikz plots
\usepackage{tikz}
\usepackage{tikzscale}
\usetikzlibrary{plotmarks,patterns,decorations.pathreplacing,backgrounds,calc,arrows,arrows.meta,spy,matrix,backgrounds}
\newcommand{\tikzmark}[1]{\tikz[overlay,remember picture] \UE (#1) {};}
\newcommand{\DrawBox}[4][]{%
    \tikz[overlay,remember picture]{%
        \coordinate (TopLeft)     at ($(#2)+(-0.4em,1.6em)$);
        \coordinate (BottomRight) at ($(#3)+(0.4em,-1.0em)$);
        %
        \path (TopLeft); \pgfgetlastxy{\XCoord}{\IgnoreCoord};
        \path (BottomRight); \pgfgetlastxy{\IgnoreCoord}{\YCoord};
        \coordinate (LabelPoint) at ($(\XCoord,\YCoord)!0.5!(BottomRight)$);
        %
        \draw [red,#1] (TopLeft) rectangle (BottomRight);
        \UE [below, #1, fill=none, fill opacity=1] at (LabelPoint) {#4};
    }
}
\usepackage{pgfplots}
\pgfplotsset{compat=newest}
\pgfplotsset{plot coordinates/math parser=false}
\usepgfplotslibrary{patchplots,groupplots}

% OUTROS PAQUETES DE USO COMUN. HOXE EN DIA OS COMPILADORES SON TAN RAPIDOS QUE EU METO TODOS SEMPRE
% \usepackage{float}
% \usepackage{ucs} 
% \usepackage{subcaption}
\usepackage{psfrag}
\usepackage{verbatim}
\usepackage{amsmath}
\usepackage{amsfonts} 
\usepackage{amssymb} 
\usepackage{amsthm}
\usepackage{pifont}
\usepackage{array}
\usepackage{listings}
\usepackage{stfloats}
\usepackage{algorithm} 
\usepackage{algorithmic} 
\usepackage{url} 
\usepackage{enumerate}
\usepackage{multirow}
\usepackage{wasysym}

\input{EETbeamerconfig.tex}
\input{mydefinitions.tex}

%---------------
% LIMIAR
%---------------
%configuracion de opcions de beamer persoais, pero alleas ao estilo

% COMANDO QUE INTRODUCE UNHA DIAPOSITIVA CUN ÍNDICE NO QUE APARECEN VELADAS TÓDALAS SECCIÓNS MENOS A ACTUAL. ÚTIL PARA INTRODUCIR OS TÍPICOS ÍNDICES INTERMEDIOS.
\newcommand{\Inter}{\frame{\tableofcontents[currentsection]}}
\newcommand{\inter}{\frame{\tableofcontents[currentsection,currentsubsection]}}

% Pes de imaxe
\renewcommand{\figurename}{Fig.}
\addto\captionsenglish{\renewcommand{\figurename}{Fig.}}
\setbeamertemplate{caption}[numbered]

%ESTE PAQUETE PERMITE POÑER A BIBLIOGRAFIA AO PE DE PAXINA CON CONFIGURACIONS ESTETICAS PERSOAIS
% \usepackage[style=ieee,doi=false,isbn=false,url=true,backend=bibtex]{biblatex}
% \bibliography{./bibliografia.bib}
% \newrobustcmd*{\footfullcitenomark}{%
%   \AtNextCite{%
%     \let\thefootnote\relax 
%     \let\mkbibfootnote\mkbibfootnotetext
%     }%
%   \footfullcite}

%paquete para engadir notas de guion ao pdf
\usepackage{pgfpages}
% \setbeameroption{show only notes} 
% \setbeameroption{show notes}
% \setbeameroption{show notes on second screen=right}
% DATOS DO DOCUMENTO
\title{Advanced Communication Systems} 
\subtitle{Part 2: Multi-user Communications}
\author[FGC]{Felipe G\'omez Cuba}
\institute[XX]{
\begin{columns}[T]
\begin{column}{9cm}\centering
Despacho 204\\
Titorías: Lun-Xov 15:00-16:30\\
(En caso de confinamento: videochamada a calquera horario acordado)\\
  \texttt{gomezcuba@gts.uvigo.es}\\
\end{column}
\end{columns}
}

\date{}

\begin{document}

% Diapositiva co título
%\frame[plain]{\titlepage}%the ``classic'' beamer cover pageç

\frame{\frametitle{\\}%generate top bar, but blank line as tittle
\titlepage
}%approximation to the ``GPSC ppt'' cover page, but with central beamer title

% \frame{\tableofcontents}
% \note[itemize]{%itemized notes are special ``note'' slides that beamer can append to the pdf or not, depending on a boolean toggle option
% \item Introduce yourself
% \item In this work we studied blablabla.
% }

\frame[allowframebreaks]{\frametitle{Presentation of Contents}
% \begin{itemize}
%  \item \textbf{Motivation}: to apply optimization and MIMO to achieve high performance multi-user communications.\\ \ \\
%  \item 
 \textbf{Outline of the Contents (15h)}:
     \begin{enumerate}
        \item Recap of single-user Information Theory from CDA (self study)
%         \begin{description}
%             \item[$\dag$] definitions: entropy
%             \item[$\dag$] usual random variables and examples 
%             \item[$\dag$] joint entropy and conditional entropy, chain rule
%             \item[$\dag$] joint entropy and vectors of random variables, typical examples
%             \item[$\dag$] diferential entropy continuous r.v.
%             \item[$\dag$] mutual information
%             \item[$\dag$] properties of entropy and mut inf. multiple variables. chain rule.
%             \item[$\dag$] notion of capacity, capacity of BSC
%             \item[$\dag$] capacity of the real AWGN channel
%             \item[$\dag$] real RF continuous time channel -> DEC -> capacity of the complex AWGN channel 
%             \item[$\dag$] lessons of the capacity proof: error pobability as low as desired as codeword size -> inf, sup Mut Inf over space of all pdfs, Gaussian achieves
%             \item[$\dag$] importance of the ML detector
%             \item[$\dag$] bits per channel use vs bps, bandwidth, spectral efficiency, energy efficiency, DoF...
%             %___________________________________s2
%             \item[$\dag$] mutual information of N-QAM, AMC to approximate capacity in practice, ML with QAM (decisor in SISO, sphere decoding in MIMO)
%             \item[$\dag$] suboptimal detectors in MIMO
%             \item[$\dag$] fading, OFDM with per-subcarrier MIMO
%             \item[$\dag$] undefined capacity in the classic sense, ergodic (emphasis OFDM) and outage (emphasis narrowband) capacity
%          \end{description}
        
        \item Multiple Access Channel and receiver (3.5h)
%         \begin{description}
%             \item[$\dag$]  concept of multi-dimensional rate vector
%             \item[$\dag$]  representing the newtwork in a graph
%             \item[$\dag$]  cut set bound method
%             \item[$\dag$]  concept of rate-region, via union of all cut set bounds
%             \item[$\dag$]  one-user cut is known capacity result
%             \item[$\dag$]  each multi user cut stems from a chain rule 
%______________________________s2
%             \item[$\dag$]  reminder of MIMO capacity and comparison of MIMO channel with cut-set bound.
%             \item[$\dag$]  achievable rate regions based on orthogonal resource division: TDMA, FDMA, CDMA and OFDMA
%             \item[$\dag$]  fixed instantaneous power vs. average power constraint vs resource allocation.
%             \item[$\dag$]  capacity achieving scheme based on non-orthogonal detection
%             \item[$\dag$]  concept of successive decoding (TIN) as interpretation of the chain rule
%             \item[$\dag$]  slow Fading, joint outage analysis and wideband optimal orthogonal
%             \item[$\dag$]  Fast Fading, averaging the cut set bound and FDMA no longer has max sum rate
%             %_____________________________________s3
%             \item[$\dag$]  Concept of individual and joint error probability
%             \item[$\dag$]  in orthogonal cases each user signal is independent and single-user decoders are ``optimal''
%             \item[$\dag$]  multiuser is harder than MIMO because of no joint encoding and asymmetries
%             \item[$\dag$]  joint MAP detector minimizes Pe
%             \item[$\dag$]  ML detector 
%             \item[$\dag$]  linear MUD [MMSE, ZF, MF]
%             \item[$\dag$]  practical SIC with QAM and independent user detector is suboptimal but faster
%             \item[$\dag$]  non-linear MUD: DF as a non-linear addition to matrix inversion Taylor method
%             \item[$\dag$]  non-linear MUD: IDD and MAP decoding schemes
%         \end{description}
        
        \item Broadcast Channel and transmitter (3.5h)
%         \begin{description}
%             \item[$\dag$] Common vs individual messages: tv broadcast vs cellular
%             \item[$\dag$] SISO, concept of a shared power constraint
%             \item[$\dag$] Single user cut bounds with full power
%             \item[$\dag$] Corner case: shared rate achievable by both users
%             \item[$\dag$] Extension to superposition code and SIC, interpreted as $P_1$ common message + $P_2$ dedicated
%             \item[$\dag$] Explanation of NOMA
%             \item[$\dag$] Dual MAC channel Theorems
%             \item[$\dag$] Interpretation of the BC capacity region as a convex optimization
%             \item[$\dag$] Interpretation as convex-hull of dual MAC channels
%             \item[$\dag$] MIMO special case with $n_t>U$, orthogonal channels. Motivation of precoding.
%             \item[$\dag$] MIMO case and dual MAC channel
%             \item[$\dag$] DoF is min(N,K)
%             \item[$\dag$] Review of dual MAC channel Theorems
%             \item[$\dag$] MMSE precoding vectors of dual MAC receiver
%             \item[$\dag$] Fixed the mmse precoding, we can modify the set of SINRs with power allocation.
%             \item[$\dag$] The power constraints of the dual are not the same as the actual allocated power
%             \item[$\dag$] Beyond linear schemes, Tomlinson-Harashima BPSK example
%             \item[$\dag$] Dirty Paper Coding for MIMO BC
%             \item[$\dag$] multiple receive antennas case
%         \end{description}
        %SISO, concept of a shared power constraint and convex-hull of dual MAC channels
        %
        \item Other multi-user topologies (2h)
            \begin{itemize}
                \item Interference Channel
%                     \begin{description}
%                         \item[$\dag$] Model of the IC
%                         \item[$\dag$] Example: cellular system with orthogonal allocation and non-cooperative BS
%                         \item[$\dag$] Very weak interference region optimal TIN
%                         \item[$\dag$] Very strong interference region optimal SIC
%                         \item[$\dag$] Interference side information and DPC
%                     \end{description}
                \item Relay Channel and multi-hop
%                     \begin{description}
%                         \item[$\dag$] Model of the RC
%                         \item[$\dag$] Cut set bounds of the RC
%                         \item[$\dag$] Degraded gaussian RC and basic two hop
%                         \item[$\dag$] DF, AF and QF protocols1
%                         \item[$\dag$] Teaser of multiuser relay channel, multi-hop graphs, etc. via cut-set combinatorial examples
%                     \end{description}
            \end{itemize}
        \item Medium Access Control (2h)
            \begin{itemize}
                \item Deterministic Access
%                     \begin{description}
%                         \item[$\dag$] Fixed vs dynamic allocation schemes
%                         \item[$\dag$] Multi-user diversity
%                         \item[$\dag$] MU-MIMO admission control
%                     \end{description}
                \item Random Access
%                     \begin{description}
%                         \item[$\dag$] ALOHA and derivates, modern collision recovery
%                         \item[$\dag$] CSMA/CA and virtual CS
%                         \item[$\dag$] Using different random timers to achieve MU-diversity and QoS too
%                     \end{description}
                \item Hybrid Access and HARQ
%                     \begin{description}
%                         \item[$\dag$] Diversity decoding with same codebook
%                         \item[$\dag$] Independent codebook approximation with puncturing
%                     \end{description}
            \end{itemize}
        \item Spectrum Co-existence (3h)
            \begin{itemize}
             \item Cognitive Radio
%              \item Sensing
             \item Network Slicing
            \end{itemize}
        \item Current systems, 5G and WiFi6 (1h)
%             \begin{itemize}
%                 \item 5G Talk By Luis
%                 \item Some pointers to WiFi6 features
%             \end{itemize}
    \end{enumerate}
% \end{itemize}




}

\frame{
\frametitle{Bibliography}

Main Reference
\begin{enumerate}
 \item David Tse, Pramod Viswanath, Fundamentals of Wireless Communication, Cambridge University Press, 2005\\
 \url{https://web.stanford.edu/~dntse/wireless\_book.html}\\ \ \\
\end{enumerate} 

For curious readers [Not required to buy!] 
\begin{enumerate}
 \item Thomas M. Cover, Joy A. Thomas, Elements of Information Theory, John Wiley \& Sons Inc 2006 
 \item Andrea Goldsmith, Wireless Communications, Cambridge University Press 2009
\end{enumerate} 

}

\end{document}


